%\textcolor{red}{MY ALTERNATE INTRO: Nonlinear and parametric interactions in the condensed matter systems provide a unique testbed on which to study ground state material properties as well as material behaviour when driven strongly out of equilibrium.   In system with coupled internal degrees of freedom, complex nonlinear interactions provide a manner by which to visualize coupled responses, and opens new possibilities for nontrivial control of a varietry of properties.  This is one of the driving forces in recent research into multiferroic materials which are expected to revolutionize electronics, spintronics, and straintronics in the coming years.  Measurements that specifically target nonlinear interactions of various degrees of freedom are thus crucial for extending our theoretical understanding of the underlying physics, while simultaneously demonstrating the range of possible interaction, and their strengths. }

%\textcolor{red}{Of particular interest are possibilities to control the magnetic properties of materials both at reduced magnetic fields and potentially via nonlinear interactions, the former resulting in reduced power requirements for writing magnetic bits while the latter provides opportunities for coherent manipulation of magnetic orientation at ever higher frequencies.  In light of these opportunities, several recent and historic studies have focused on the nonlinear and parametric driving of magnetic precession using a wide range of techniques including seeded parametric excitation of magnetization precession by radio frequency excitation\cite{elazzabi},XXX, and magnetoelastic interactions. In this last case, elasticity and coherent elastic deformations interact with the magnetization in a complex manner, driving precession at fractions and multiples of underlying elastic frequencies. }

%\textcolor{red}{In this report, we extend our recent demonstrations of linear magnetoelastic coupling and show the magnetization dynamics in thin metal films parametrically driven by single or multiple elastic waves.  The resultant magnetization dynamics exhibit a range of precessional modes, at harmonics of the underlying elastic waves as wellas at half frequencies.  The behaviour is modeled in the framework of parametric excitation and we demonstrate excellent correspondence between our numerical model and experimental observations.}

%\textcolor{red}{The interaction between material elasticity and magnetization has recently attracted attention both as a means of efficient control at Gigahertz frequencies\cite{Bombeck} as well as low frequency 'straintronic' control\cite{Thevenard3, Roy} (for a complete list of recent experiments see the reference in \cite{Janusonis2015, Janusonis2016, Janusonis2016_1}).  In light of these studies, we implemented a simple method\cite{Janusonis2015, Janusonis2016, Janusonis2016_1} by which perform magnetoelastic control experiments that utilizes the all- optical transient grating (TG) geometry, wherein the elastic and magnetic degrees of freedom can be monitored independently and their coupling can be tuned by the external application of a magnetic field. The elastic dynamics can further be engineered by appropriate choice of substrate material, allowing to study the magnetization response when driven by either a single strong elastic wave or a combination of elastic modes.  This variability in elastic dynamics provides for an unequivocal determination of sum and difference frequency generation in the magnetization precession, as well as comparison with theory.   Our initial demonstration opens the door to more complex elastic wave\cite{Schulein} control over magnetization  that could result in extremely broadband, and widely tunable, control of magnetization precession.}


%,ChengAPL2013,KeshtgarSSC2014}. Here, the continuous excitation of the FMR by multiple high power radio frequency excitation has shown the excitation of second harmonic generation \cite{ElezzabiAPL2003,} as well as seeded parametric downconversion, in effect, leading to a range of frequency conversion techniques.

%\textcolor{red}{also cite Hillebrands}.\textcolor{red}{also cite Hillebrands}.

%and large-amplitude nonlinear precession \cite{GerritsPRL2007}.

%For example parametrically driven magnetization precession showing %nonlinear precession generation was demonstrated in garnet films %\cite{ElezzabiAPL2003} and more recently in thin films of %CoFeB\cite{CapuaPRL2016}, both utilizing radio frequency excitation of %magnetic dynamics. 
%, acoustic, and magnetic subsystems often couple via nonlinear channels \cite{TemnovJOPT2016} albeit, \textcolor{red}{I dont' think this makes much sense: given that the underlying nonlinearities are instrinsically different, occur at very different time and energy scales.} While it is usually difficult to modulate  optical nonlinearities on a sub-cycle (femtosecond) time scale, in magnetism and acoustics the task is more straightforward. Indeed, significantly lower FMR-frequencies in magnetism and vibrational frequencies of acoustic phonons in solids often fall within the MHz-GHz spectral range resulting in opportunities for their resonant coupling, and specificlaly opportunities to study their nonlinear interactions. 


%In parallel efforts, elastic coupling to magnetization is also seen as a means to achieve magnetization control without the required magnetic fields, for example in elastic reduction of coercivity\cite{Thevenard3} or direct reversal of the magnetization direction\cite{Kovalenko}. Extending these effects to ever higher frequencies opens the door to controllable, multi-directional elastic actuation of material magnetization in the multi-Gigahertz regime and heralding efficient structural control of magnetism. 



%Along with using femtosecond time resolved measurement techniques \cite{CapuaPRL2016} to monitor FMR precession a straightforward ways to extend the spectral range higher is to implement frequency mixing, which requires multiple, possibly different, monochromatic excitations to act on the magnetic subsystem. 

%\textcolor{red}{I WOULD LEAVE THIS OUT - {\it Agreed. It will go to the supplementary.} Taking into account that the elastic field associated with SSLW varies in time we will focus on the magneto-elastic nonlinearities induced by SAW, which amplitude is constant within 8~ns time window of Faraday rotation measurement \cite{Janusonis2016_2}.  Understanding of SAW second harmonic signal at 5~GHz frequency represents the main goal of this paper.} 

%Extending these studies now allows us to uniquely identify nonlinear magnetoelastic behavior in the simple metal-on-glass heterostructure. Specifically, the magnetization precession frequency appears as the sum of the two elastic wave frequencies, thus demonstrating nonlinearities in the magnetoelastic coupling, mimicking nonlinear optical phenomena such as sum frequency generation, and 

%while the coupling strength between the elastic deformation and the magnetization could also be tuned by adjusting the angle between the static magnetization and the optically excited wavevector of the elastic wave. 

%Magnetoelasticity couples structural deformations to the magnetic properties in materials, and provides the possibility of actively controlling magnetization dynamics into the tens of Gigahertz regime and beyond.  In the low GHz frequency range, this has been demonstrated as an efficient manner in which to pump spins for spintronics applications, while at quasi-DC frequencies it has been proposed as a low energy alternative for switching magnetization in nanomagnets.  As frequencies increase, the deformation amplitudes can also increased, and the possibility for nonlinear coupling between the two degress of freedom become possible, opening the door to control capabilities that have heretofore been inaccessible.  Here, we demonstrate the first instance of nonlinearities in magnetoelastic coupling. We show the generation of magnetoelastic waves that are the sum frequency of the underlying elastic excitations and support these findings by full calculations of the Landau-Lifshitz-Gilbert equation where multiple elastic waves are allowed to interact nonlinearly with the material magnetization.






%For ease of display, each individual plot is scaled from zero to 100, and then plotted in logarithmic scale. In both data and simulation, amplitudes of the nonlinear response at $\phi < 10^{\circ}$ are at most $1\%$ of the linear response amplitude.


%each detection channel is independently sensitive to its respective degree of freedom, which provides for the unique detection of average magnetization dynamics in the metallic film, subject to the excitation of elastic waves propagating along the sample surface and the temperature profile induced by the spatial periodicity of the excitation.  

%\begin{figure}
%    \centering
%    \includegraphics[width=\columnwidth]{TGBeamPath.png}
%    \includegraphics[width=\columnwidth]{TimeTraces.eps}
%    \caption{ (a) In the ultrafast transient grating geometry, two spatially overlapped and time-coincident pump pulses (blue) generate a spatially inhomogeneous excitation profile on the sample surface, launching surface propagating acoustic waves. Coincidence is ensured by imaging a binary phase mask onto the sample. A time delayed, normally incident, probe pulse is polarization analyzed (Faraday detection) to provide information on the average magnetization. Either separately or simultaneously, diffraction of a probe beam (dashed line) measures the elastic deformations associated with the surface propagating acoustic waves.  (b) Time resolved Faraday effect is measured as a function of applied external magnetic field and unveils the resonant interaction between elastic and magnetic degrees of freedom.}
%    \label{fig:dispersion}
%\end{figure}


%The data shown here is for the excitation grating period of $\Lambda = 1.7\mu$m.  In all cases barring the lowest applied field, quasi-monochromatic oscillations can be seen at each of the resonance conditions.  At the lowest field, combinations of SAW and SSLW appear simultaneously in the time trace making it difficult to isolate the resonant precession due to individual applied strains.

%In figure 1b we show representative scans of magnetization precession at various applied fields as we tune through the resonances of SAW (52G) at 1.8GHz, SSLW (193G) at 3.5GHz.  Upon further increase of the applied field, additional resonances are witnessed that match the frequencies of SAW + SSLW (500G) at 5.3GHz and SSLW + SSLW (822G) at 7GHz. 





