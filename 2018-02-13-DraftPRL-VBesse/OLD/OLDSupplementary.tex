%\documentclass{report}
\documentclass[preprint,superscriptaddress]{revtex4}
\usepackage{graphicx}
%\usepackage{grffile}
\usepackage{amsmath}
\usepackage{textcomp}
\usepackage{color}
\begin{document}

\title{Supplemental Materials for Elastically Activated Nonlinear Magnetization Precession}

\author{C.L. Chang}
\thanks{These authors contributed equally}
\affiliation{Zernike Institute for Advanced Materials, University of Groningen, Groningen, The Netherlands}
\author{V.S. Vlasov}
\thanks{These authors contributed equally}
\affiliation{IMMM CNRS 6283, Universit\'e du Maine, 72085 Le Mans cedex, France}
\affiliation{Syktyvkar State University named after Pitirim Sorokin, 167001, Syktyvkar, Russia}
\author{J. Janusonis}
\affiliation{Zernike Institute for Advanced Materials, University of Groningen, Groningen, The Netherlands}
\author{A.M. Lomonosov}
\affiliation{LAUM CNRS 6613, Universit\'e du Maine, 72085 Le Mans cedex, France}
\author{V.V. Temnov}
\affiliation{IMMM CNRS 6283, Universit\'e du Maine, 72085 Le Mans cedex, France}
\affiliation{Fritz-Haber-Institut der Max-Planck-Gesellschaft, Abteilung Physikalische Chemie, Faradayweg 4-6, 14195 Berlin, Germany}
\author{R.I. Tobey}
\email{r.i.tobey@rug.nl}
\affiliation{Zernike Institute for Advanced Materials, University of Groningen, Groningen, The Netherlands}
\maketitle
{\bf SUPPLEMENTAL MATERIAL}

The numerical solution based on the calculated Green's
function, as we have discussed previously\cite{Janusonis2016}, show that 
in-plane longitudinal tensile/compressive strain $\epsilon_{xx}$ is the largest strain amplitude, which 
is the only component active in the SSLW wave and the stronger component in the 
Rayleigh SAW wave (see also Dreher et.al.\cite{Dreher}).
We take into account only main strain component in SAW $\epsilon_{xx}(t)$. The time-dependent stress $\epsilon_{xx}(t)$ can
be fitted by a superposition of two contributions, SAW
($\sim\cos(2\pi f_{SAW}t)$) and SSLW
($\sim\cos(2\pi f_{SSLW}t+\psi)/t$).


In order to describe the magnetization dynamics in nickel induced
by SAW and SSLW waves we derive the equations for FMR-precession
from the Landau-Lifshitz-Gilbert equation. The phenomenological
expression for free energy density
$F(\overrightarrow{M})=F_{me}+F_d+F_z$ for a thin polycrystalline
thin film of Nickel reads \cite{delaFuente04JPCM16}:
\begin{eqnarray}
\label{me-term}
F_{me}&=&b_1m^2_x\epsilon_{xx}\,\\
\label{ms-term} F_d&=&\frac{\mu_0}{2}(M_0 m_z)^2\\
\label{Zeeman-term}
F_Z&=&-{\mu_0}\overrightarrow{M}\overrightarrow{H}\,,
\end{eqnarray}
where $m_x,m_y,m_z$ are components of unit magnetization vector
$\overrightarrow{m}=\overrightarrow{M}/M_0$ in the
crystallographic coordinate system $(x,y,z)$ and saturation
magnetization $\mu_0M_0=0.6$~T. In the experimental geometry with
collinear magnetic field $\overrightarrow{H}$ and transient
grating wave vector pointing in $x$-direction there are one main
component of acoustic strain in SAW acting on
magnetization in a thin layer of nickel:
\begin{eqnarray}
\label{SAW_strain}
\epsilon^{SAW}_{xx}(t)&=&\epsilon_{xx1}\cos(kx)\cos(\omega_{SAW}t).
\end{eqnarray}
In contrast, the surface skimming longitudinal wave (SSLW) is
dominated by a single strain component
$\epsilon^{SSLW}_{xx}(t)=\epsilon_{xx2}\cos(kx)\cos(\omega_{SSLW}t)/t$.

The dynamics of magnetization are described by the
Landau-Lifshitz-Gilbert (LLG) eguation
\begin{equation}\label{LLGEquation}
\frac{d\overrightarrow{M}}{dt}=-\gamma\mu_0
(\overrightarrow{M}\times\overrightarrow{H}_{eff})
+\frac{\alpha}{M_s} \bigg (\overrightarrow{M}\times
\frac{d\overrightarrow{M}}{dt}\bigg )
\end{equation}
driven by a time-dependent effective magnetic field
\begin{equation}
\label{Heff}
\overrightarrow{H}_{eff}(t)=-\frac{1}{\mu_0}\frac{dF(t)}{d\overrightarrow{M}}\\.
\end{equation}
The components of the effective magnetic field:
\begin{eqnarray}
\label{Heff_comp}
H_{eff,x}&=&\frac{1}{M_0\mu_0}2b_1\epsilon_{xx}b_1+H\cos\phi  \label{Heff_comp1}
\\
H_{eff,y}&=&H\sin\phi \label{Heff_comp2}
\\
H_{eff,z}&=&-M_0m_z \label{Heff_comp3}
\,.
\end{eqnarray}
where $\epsilon_{xx}=\epsilon^{SAW}_{xx}(t)+\epsilon^{SSLW}_{xx}(t)$,
$\phi$ is the angle between the magnetic field and
transient grating wave vector.
The vector equation  (\ref{LLGEquation}) can be reduced to the system of equations:

\begin{eqnarray}
\frac{d m_x}{dt}=-\frac{\gamma\mu_0}{1+\alpha^{2}}
((m_y+\alpha m_x m_z) H_{eff,z}-\nonumber\\
- (m_z - \alpha m_y m_x)  H_{eff,y}- \alpha (m_y^2+m_z^2)  H_{eff,x}) \label{LLG_comp1}
\\
\frac{d m_y}{dt}=-\frac{\gamma\mu_0}{1+\alpha^{2}}
((m_z+\alpha m_y m_x) H_{eff,x}-\nonumber\\
- (m_x - \alpha m_z m_y)  H_{eff,z}- \alpha (m_z^2+m_x^2)  H_{eff,y}) \label{LLG_comp2}
\\
\frac{d m_z}{dt}=-\frac{\gamma\mu_0}{1+\alpha^{2}}
((m_x+\alpha m_z m_y) H_{eff,y}-\nonumber\\
- (m_y - \alpha m_x m_z)  H_{eff,x}- \alpha (m_x^2+m_y^2)  H_{eff,z}) \label{LLG_comp3}
\,
\end{eqnarray}

\textcolor{red}{This is the align environment that might be a little cleaner:}
\begin{align}
\frac{d m_x}{dt}=&-\frac{\gamma\mu_0}{1+\alpha^{2}}
((m_y+\alpha m_x m_z) H_{eff,z}-\nonumber\\
&- (m_z - \alpha m_y m_x)  H_{eff,y}- \alpha (m_y^2+m_z^2)  H_{eff,x}) \label{LLG_comp1}
\\
\frac{d m_y}{dt}=&-\frac{\gamma\mu_0}{1+\alpha^{2}}
((m_z+\alpha m_y m_x) H_{eff,x}-\nonumber\\
&- (m_x - \alpha m_z m_y)  H_{eff,z}- \alpha (m_z^2+m_x^2)  H_{eff,y}) \label{LLG_comp2}
\\
\frac{d m_z}{dt}=&-\frac{\gamma\mu_0}{1+\alpha^{2}}
((m_x+\alpha m_z m_y) H_{eff,y}-\nonumber\\
&- (m_y - \alpha m_x m_z)  H_{eff,x}- \alpha (m_x^2+m_y^2)  H_{eff,z}) \label{LLG_comp3}
\
\end{align}


We substitute the equations (\ref{Heff_comp1})-(\ref{Heff_comp3}) to the equations (\ref{LLG_comp1})-(\ref{LLG_comp3}) 
and have the system of nonlinear differential equations. For solution of the system of the equations we will use 
the Runge-Kutta 4-5 orders method.
The initial conditions for unit magnetization vector components correspond to the equilibrium orientation 
of the vector along DC magnetic field: $m_x(0)=\cos\phi, m_y(0)=\sin\phi, m_z(0)=0.$



\bibliography{CLChang.bib}
\end{document}
