%\documentclass{report}
\documentclass[aps,prl,amsmath,amssymb,preprint,superscriptaddress]{revtex4-1}
\usepackage{graphicx}
%\usepackage{grffile}
\usepackage{amsmath}
\usepackage{textcomp}
\usepackage{color}
\usepackage{textcomp}
\usepackage{gensymb}
\usepackage{multirow}

\usepackage{array}
\newcolumntype{L}[1]{>{\raggedright\let\newline\\\arraybackslash\hspace{0pt}}m{#1}}
\newcolumntype{C}[1]{>{\centering\let\newline\\\arraybackslash\hspace{0pt}}m{#1}}
\newcolumntype{R}[1]{>{\raggedleft\let\newline\\\arraybackslash\hspace{0pt}}m{#1}}

\begin{document}

\title{Exchange Magnons in Ferromagnetic Films Excited by Picosecond Acoustic Pulses\\ Supplementary Information}

\author{V. Besse}
\email{valentin.besse@univ-lemans.fr}
\affiliation{IMMM CNRS 6283, Le Mans Universit\'{e}, 72085 Le Mans cedex, France}
\author{A.V. Golov}
\affiliation{Syktyvkar State University named after Pitirim Sorokin, 167001, Syktyvkar, Russia}
\author{V.S. Vlasov}
\affiliation{Syktyvkar State University named after Pitirim Sorokin, 167001, Syktyvkar, Russia}
\author{L.N. Kotov}
\affiliation{Syktyvkar State University named after Pitirim Sorokin, 167001, Syktyvkar, Russia}
\author{A. Alekhin}
\affiliation{IMMM CNRS 6283, Le Mans Universit\'{e}, 72085 Le Mans cedex, France}
\author{}
\author{V.V. Temnov}
\affiliation{IMMM CNRS 6283, Le Mans Universit\'{e}, 72085 Le Mans cedex, France}

\maketitle
%{\bf SUPPLEMENTAL MATERIAL}

The supplementary information is split in ... sections.
The first section is dedicated to the calculation of the model we numerically resolve in the main text.

\section{Magnetization dynamics excited by acoustic pulses}
\label{sec:SI1}

We consider the Landau-Lifschitz-Gilbert equation in the Cartesian coordinate system $\left(x,y,z\right)$
\begin{equation}
    \frac{\partial \mathbf{m}}{\partial t} = - \gamma \mu_0 \mathbf{m} \times \mathbf{H}_{eff} + \alpha \mathbf{m} \times \frac{\partial \mathbf{m}}{\partial t},
    \label{eq:LLG-1}
\end{equation}
which describe the magnetization dynamics in a ferromagnetic film, where $\mathbf{m}$ is the unit magnetization vector, $\gamma$ is the gyromagnetic ratio, $\mu_0$ is the vacuum permeability and $\mathbf{H}_{eff}$ is the effective magnetic field.
It is the functional derivative of the free energy density
\begin{equation}
    \mathbf{H}_{eff} = - \frac{1}{\mu_0 M_0} \frac{\partial U}{\partial \mathbf{m}} + \frac{1}{M_0} \sum_{p=1}^{3} \frac{\partial}{\partial x_p} \frac{\partial U}{\partial \left(\frac{\partial  \mathbf{m}}{\partial x_p} \right)},
    \label{eq:Heff}
\end{equation}
where $M_0$ is the magnetic saturation.
We define the free density energy to be 
\begin{equation}
    U = U_{de} + U_{z} + U_{me} + U_{ex},
    \label{eq:U}
\end{equation}
which include the demagnetizing field's energy
\begin{equation}
    U_{de} = \frac{1}{2} \mu_0 M_0^2 \mathbf{m} \cdot \mathbf{N} \cdot \mathbf{m},
    \label{eq:Ude}
\end{equation}
the external magnetic field's energy
\begin{equation}
    U_z = -\mu_0 M_0 \mathbf{m} \cdot \mathbf{H},
    \label{eq:Uz}
\end{equation}
the magnetoelastic field energy
\begin{equation}
    U_{me} = b_1 \sum_{p=1}^3 m_p^2\epsilon_{pp}
    \label{eq:Ume}
\end{equation}
and the exchange magnetic field energy
\begin{equation}
    U_{ex} = \frac{1}{2} M_0 \sum_{p=1}^3 D \left( \frac{\partial \mathbf{m}}{\partial x_p} \right)^2.
    \label{eq:Uex}
\end{equation}
The magnetoelastic constant is $b_1$, while the exchange constant is $D$.

We consider the situation shown in the figure 1 of the main text.
It implies that the unit magnetic vector $\mathbf{m}$ and the external magnetic vector $\mathbf{H}$ are in the plane $\left(x O z\right)$.
The consequence is that the demagnetization tensor is
\begin{equation}
    \mathbf{N} = \begin{pmatrix} 0 & 0 & 0 \\
    0 & 0 & 0 \\
    0 & 0 & 1
    \end{pmatrix},
    \label{eq:N}
\end{equation}
and the acoustic strain is
\begin{equation}
    \sum_{p=1}^3\epsilon_{pp} = \epsilon_{zz}.
    \label{eq:epsilon}
\end{equation}


First, we consider that $\alpha = 0$, so we can rewrite \ref{eq:LLG-1}:
\begin{equation}
    \frac{\partial \mathbf{m}}{\partial t} = - \gamma \mu_0 \mathbf{m} \times \mathbf{H}_{eff},
    \label{eq:LLG-2}
\end{equation}
%\bibliography{bib_ExchangeMagnons.bib}

\section{Strain dependence of exchange field}

\section{Parameters used in the simulation}

The relevant parameters are given in Table \ref{table:parameters}.

\begin{table}[ht]
    \centering
    \begin{tabular}{C{3cm}|C{5cm}|C{4cm}}
    \hline
    \hline
    Group & Parameter & Value \\
    \hline
    \multirow{6}{*}{Nickel} & Thickness & $30\ \mathrm{nm}$\\
    & H-field & $484.1 \times 10^{3}\ \mathrm{A}/\mathrm{m}$\\
    & Sound velocity $c_{\mathrm{s}}$ & $5.6\times 10^3\ \mathrm{m}/\mathrm{s}$\\
    & Exchange constant $D$ & $430\ \mathrm{meV.A}^2$\\
    & Magnetoelastic constant $b_1$ & $1\times 10^7$\\
    & Electron gyromagnetic ratio $\gamma$& $1.76 \times 10^{11}\ \mathrm{rad}/\mathrm{s.T}$\\
    \hline
    \multirow{3}{*}{Adjustable}& Magnetic field angle $\xi$ & $45\mathrm{\degree}$ \\
    & Acoustic pulse duration & $1\ \mathrm{ps}$ \\
    & Maximal propagation time & $400\ \mathrm{ps}$ \\
    \hline
    \hline
    \end{tabular}
    \caption{Parameter’s value used for the numerical simulation}
    \label{table:parameters}
\end{table}

\end{document}
