%\documentclass{report}
\documentclass[aps,prl,amsmath,amssymb,preprint,superscriptaddress]{revtex4-1}
\usepackage{graphicx}
%\usepackage{grffile}
\usepackage{amsmath}
\usepackage{textcomp}
\usepackage{color}
\usepackage{textcomp}
\usepackage{gensymb}
\usepackage{multirow}
\usepackage{array}
\usepackage{cases}

\newcolumntype{L}[1]{>{\raggedright\let\newline\\\arraybackslash\hspace{0pt}}m{#1}}
\newcolumntype{C}[1]{>{\centering\let\newline\\\arraybackslash\hspace{0pt}}m{#1}}
\newcolumntype{R}[1]{>{\raggedleft\let\newline\\\arraybackslash\hspace{0pt}}m{#1}}

\begin{document}

\title{Exchange Magnons in Ferromagnetic Films Excited by Picosecond Acoustic Pulses\\ Supplementary Information}

\author{V. Besse}
\email{valentin.besse@univ-lemans.fr}
\affiliation{IMMM CNRS 6283, Le Mans Universit\'{e}, 72085 Le Mans cedex, France}
\author{A.V. Golov}
\affiliation{Syktyvkar State University named after Pitirim Sorokin, 167001, Syktyvkar, Russia}
\author{V.S. Vlasov}
\affiliation{Syktyvkar State University named after Pitirim Sorokin, 167001, Syktyvkar, Russia}
\author{L.N. Kotov}
\affiliation{Syktyvkar State University named after Pitirim Sorokin, 167001, Syktyvkar, Russia}
\author{A. Alekhin}
\affiliation{IMMM CNRS 6283, Le Mans Universit\'{e}, 72085 Le Mans cedex, France}
\author{}
\author{V.V. Temnov}
\affiliation{IMMM CNRS 6283, Le Mans Universit\'{e}, 72085 Le Mans cedex, France}

\maketitle
%{\bf SUPPLEMENTAL MATERIAL}

The supplementary information is split in ... sections.
The first section is dedicated to the calculation of the model we numerically resolve in the main text.

\section{Magnetization dynamics excited by acoustic pulses}
\label{sec:SI1}

We consider the Landau-Lifschitz-Gilbert equation in the Cartesian coordinate system $\left(x,y,z\right)$
\begin{equation}
    \frac{\partial \mathbf{m}}{\partial t} = - \gamma \mu_0 \mathbf{m} \times \mathbf{H}_{eff} + \alpha \mathbf{m} \times \frac{\partial \mathbf{m}}{\partial t},
    \label{eq:LLG-1}
\end{equation}
which describe the magnetization dynamics in a ferromagnetic film, where $\mathbf{m}$ is the unit magnetization vector, $\gamma$ is the gyromagnetic ratio, $\mu_0$ is the vacuum permeability and $\mathbf{H}_{eff}$ is the effective magnetic field.
It is the functional derivative of the free energy density
\begin{equation}
    \mathbf{H}_{eff} = - \frac{1}{\mu_0 M_0} \frac{\partial U}{\partial \mathbf{m}} + \frac{1}{M_0} \sum_{p=1}^{3} \frac{\partial}{\partial x_p} \frac{\partial U}{\partial \left(\frac{\partial  \mathbf{m}}{\partial x_p} \right)},
    \label{eq:Heff}
\end{equation}
where $M_0$ is the magnetic saturation.
We define the free density energy to be 
\begin{equation}
    U = U_{de} + U_{z} + U_{me} + U_{ex},
    \label{eq:U}
\end{equation}
which include the demagnetizing field's energy
\begin{equation}
    U_{de} = \frac{1}{2} \mu_0 M_0^2 \mathbf{m} \cdot \mathbf{N} \cdot \mathbf{m},
    \label{eq:Ude}
\end{equation}
the external magnetic field's energy
\begin{equation}
    U_z = -\mu_0 M_0 \mathbf{m} \cdot \mathbf{H},
    \label{eq:Uz}
\end{equation}
the magnetoelastic field energy
\begin{equation}
    U_{me} = b_1 \sum_{p=1}^3 m_p^2\epsilon_{pp}
    \label{eq:Ume}
\end{equation}
and the exchange magnetic field energy
\begin{equation}
    U_{ex} = \frac{1}{2} M_0 \sum_{p=1}^3 D \left( \frac{\partial \mathbf{m}}{\partial x_p} \right)^2.
    \label{eq:Uex}
\end{equation}
The magnetoelastic constant is $b_1$, while the exchange constant is $D$.
In this section, we will consider that the exchange magnetic field energy does not depend on the strain.

We consider the situation shown in the Fig. 1 of the main text.
It implies that the unit magnetic vector $\mathbf{m}$ and the external magnetic vector $\mathbf{H}$ are in the plane $\left(x O z\right)$.
The consequence is that the demagnetization tensor is
\begin{equation}
    \mathbf{N} = \begin{pmatrix} 0 & 0 & 0 \\
    0 & 0 & 0 \\
    0 & 0 & 1
    \end{pmatrix},
    \label{eq:N}
\end{equation}
and the acoustic strain is
\begin{equation}
    \sum_{p=1}^3\epsilon_{pp} = \epsilon_{zz}.
    \label{eq:epsilon}
\end{equation}
By using Eqs. \eqref{eq:Heff}, \eqref{eq:U}, \eqref{eq:Ude}, \eqref{eq:Uz}, \eqref{eq:Ume} and \eqref{eq:Uex} and in accordance with the Fig. 1, we obtain
%\begin{numcases}{}
%    H_\mathrm{eff,x} = H \cos \left( \xi \right) - \frac{M_0}{\mu_0} D \frac{\partial m_\mathrm{x}}{\partial z} + M_0 D \frac{\partial^2 m_\mathrm{x}}{\partial z^2} \\
%    H_\mathrm{eff,y} = - \frac{M_0}{\mu_0} D \frac{\partial m_\mathrm{y}}{\partial z} + M_0 D \frac{\partial^2 m_\mathrm{y}}{\partial z^2}\\
%    H_\mathrm{eff,z} = H \sin \left( \xi \right) - \frac{2 b_1}{\mu_0 M_0} m_\mathrm{z} \epsilon_{\mathrm{zz}} - \frac{M_0}{\mu_0} D \frac{\partial m_\mathrm{z}}{\partial z} + M_0 D \frac{\partial^2 m_z}{\partial z^2} - M_0 \mathrm{m_\mathrm{z}}
%\end{numcases}
\begin{numcases}{}
    H_\mathrm{eff,x} = H \cos \left( \xi \right) + M_0 D \frac{\partial^2 m_\mathrm{x}}{\partial z^2} \\
    H_\mathrm{eff,y} = M_0 D \frac{\partial^2 m_\mathrm{y}}{\partial z^2}\\
    H_\mathrm{eff,z} = H \sin \left( \xi \right) - \frac{2 b_1}{\mu_0 M_0} m_\mathrm{z} \epsilon_{\mathrm{zz}} + M_0 D \frac{\partial^2 m_z}{\partial z^2} - M_0 \mathrm{m_\mathrm{z}}
\end{numcases}

First, we consider that $\alpha = 0$, so we can rewrite \ref{eq:LLG-1}:
\begin{equation}
    \frac{\partial \mathbf{m}}{\partial t} = - \gamma \mu_0 \mathbf{m} \times \mathbf{H}_{eff}.
    \label{eq:LLG-2}
\end{equation}
We represent the unit magnetization vector as a Fourier series
\begin{equation}
    \mathbf{m} = \mathbf{m}^0 + \sum_{\mathrm{n=0}}^\mathrm{N} \mathbf{m}_{\mathrm{n}} \left( t \right) \cos \left( \frac{\pi n}{L} z \right),
    \label{eq:mFourier}
\end{equation}
where $\mathbf{m}^0$ is the stable state equilibrium without any perturbation, n is the magnon mode number, $L$ is the thickness of ferromagnetic film. 
The constant component of the magnetization vector tilted by a $\theta$ angle with respect to the $x$ axis (see Fig. 1 in the main manuscript).
We set the boundary condition to
\begin{equation}
    \left. \frac{\partial m_i}{ \partial z} \right|_{z = 0,L} =0.
\end{equation}

\section{Magnetoelastic field}


\section{Strain dependence of exchange field}

In this section, we will consider that the exchange field energy is a strain dependent process

The energy of the exchange interaction in a ferromagnet can be represented by \cite{gurevich1996magnetization}
\begin{equation}
    U_{\mathrm{ex}} = U_{\mathrm{ex,0}} + U_{\mathrm{ex,NU}}
    \label{eq1}
\end{equation}
where $U_{\textrm{ex,0}}$ is the exchange energy when the magnetization is uniform and $U_{\textrm{ex,NU}}$ is responsible for the change of the exchange energy when the magnetization is non-uniform.
The first term of Eq. \eqref{eq1} can be written
\begin{equation}
    U_{\mathrm{ex,0}} = \frac{1}{2} \mathbf{M} \overleftrightarrow{\Lambda} \mathbf{M}
    \label{eq2}
\end{equation}
with the magnetization $\mathbf{M}$ and the exchange tensor $\overleftrightarrow{\Lambda}$.

The second term of Eq. \eqref{eq1} can be written
\begin{equation}
    U_{\mathrm{ex,NU}} = \frac{1}{2} \sum_{p=1}^3 \sum_{s=1}^3 q_{ps} \frac{\partial \mathbf{M}}{\partial x_p} \frac{\partial \mathbf{M}}{\partial x_s}
    \label{eq3}
\end{equation}
where $q_{ps}$ are the components of a tensor $\overleftrightarrow{q}$.

The effective field is a functional derivative (or variational derivative) of the energy by the magnetization vector
\begin{equation}
    H_{\mathbf{eff}} = - \frac{\delta U}{\delta \mathbf{M}} = - \frac{\partial U}{\partial \mathbf{M}} + \sum_{p=1}^3 \frac{\partial}{\partial x_p} \left[ \frac{\partial U}{\partial \left(\partial \mathbf{M} / \partial x_{p} \right)} \right]
    \label{eq4}
\end{equation}
with the total free energy $U$.
Using Eqs. \eqref{eq3} and \eqref{eq4} we obtain the contribution of the exchange interaction to the effective field
\begin{equation}
    \mathbf{H}_{\mathrm{ex}} = \mathbf{H}_{\mathrm{ex,0}} + \mathbf{H}_{\mathrm{ex,NU}} \equiv \overleftrightarrow{\Lambda} \mathbf{M} + \sum_{p=1}^3 \sum_{s=1}^3 q_{ps} \frac{\partial^2 \mathbf{M}}{\partial x_p \partial x_s}
    \label{eq5}
\end{equation}

So now, let's consider the case of an isotropic ferromagnet, it implies that $\Lambda$ and $q$ are scalars.
We write $q = D$.
We also consider the equation only for the x-direction.
The Eq. \eqref{eq5} becomes
\begin{equation}
    \mathbf{H}_{\mathrm{ex}} = \Lambda \mathbf{M} + D \frac{\partial^2 \mathbf{M}}{\partial x^2}.
    \label{eq6}
\end{equation}
We write the magnetization vector as a Fourier series
\begin{equation}
    \mathbf{M} = \mathbf{M}_0 + \sum_{n=0}^N \mathbf{M}_n \left( t \right) \cos \left( k_n x\right)
    \label{eq7}
\end{equation}
where $\mathbf{M}_0$ is the steady states of the magnetization vector, $\mathbf{M}_n$ is the n-th order of the magnetization vector and $k_n$ is the wavevector of the n-th magnetic order.
It implies that $\mathbf{M}_n \ll \mathbf{M}_0$.
Substitutes Eq. \eqref{eq7} into the non-uniform part of Eq. \eqref{eq6} we obtain 
\begin{equation}
    \mathbf{H}_{\mathrm{ex,NU}} =  D \frac{\partial^2 \mathbf{M}}{\partial x} = - D \sum_{n=0}^N k^2_n \mathbf{M}_n^2 \cos \left( k_n x\right).
    \label{eq8}
\end{equation}
The lossless Landau-Lifshitz-Gilbert equation is defined by
\begin{equation}
    \frac{\partial \mathbf{M}}{\partial t} = - \gamma \mathbf{M} \times \mathbf{H}_{\mathrm{eff}}
    \label{eq9}
\end{equation}
with $\gamma$ is the gyromagnetic ratio.
It is obvious that since $\mathbf{H}_{\mathrm{ex,0}}$ depends on $\textbf{M}$ so it does not play a role $$\mathbf{M} \times \mathbf{H}_{\mathrm{ex,0}} =0.$$
Adding Eq. \eqref{eq8} into Eq. \ref{eq9} we obtain
\begin{equation}
    \frac{\partial \mathbf{M}}{\partial t} + \gamma D \left[ \mathbf{M} \times \frac{\partial^2 \mathbf{M}}{\partial x^2} \right] = 0
    \label{eq10}
\end{equation}
Let's consider that the time dependence of the magnetization vector is $$\mathbf{M} \left(t, \mathbf{r} \right) \sim \mathbf{M} \left( \mathbf{r} \right) \exp \left( i \omega t \right)$$ and that the exchange constant $D$ is modified by the propagation of a strain: the distance between two atoms changes.
So the exchange coefficient depends on time and space.
\begin{equation}
    i \omega \mathbf{M} + \gamma D \left( t, z \right) \left[ \mathbf{M} \times \frac{\partial^2 \mathbf{M}}{\partial x^2} \right] = 0
    \label{eq11}
\end{equation}
The Eq. \eqref{eq11} shows that the exchange interaction appears only in parametric terms.
It means that this term cannot drive the magnons by itself if $\mathbf{M}_n \left( t = 0, \mathbf{r} \right) = \overrightarrow{0}$.

Now, let's consider that the exchange coefficient is dependant on strain ($\epsilon$) propagation.
We write
\begin{equation}
    D \left( t, z \right) \sim D_{0} \exp \left( -\frac{\Delta r}{R_{\mathrm{D}}} \right)
    \label{eq12}
\end{equation}
where $D_0$ is the exchange coefficient equilibrium value at the equilibrium, $\Delta r$ is the distance between two neighbor electrons which depends on the strain and $R_{\mathrm{D}}$ is the typical length of the screening effect.
The distance between two neighbor electrons change proportionally to the amplitude of the strain $$ \Delta r \sim r_0 \epsilon$$
where $r_0$ is the distance between two neighbor electrons at the equilibrium.
The Eq. \eqref{eq12} is
\begin{equation}
    D \left( t, z \right) \sim D_{0} \exp \left( -\frac{r_0 \epsilon}{R_{\mathrm{D}}} \right)
    \label{eq13}
\end{equation}
We develop in Taylor series the Eq. \eqref{eq13}
\begin{equation}
    D \left( t, z \right) \sim () D_{0} \left( 1 + \frac{r_0}{R_{\mathrm{D}}} \epsilon \right).
    \label{eq14}
\end{equation}
The typical length of the screening effect is in the range of the size of the lattice $r_0$.
Consequence Eq. \eqref{eq14} is
\begin{equation}
    D \left( t, z \right) \sim D_{0} \left( 1 + \epsilon \right).
    \label{eq15}
\end{equation}
So the first-order Taylor coefficient is in the order of the strain amplitude, so typically from $0.01 \%$ to $0.1\%$.

\section{Parameters used in the simulation}

The relevant parameters are given in Table \ref{table:parameters}.

\begin{table}[ht]
    \centering
    \begin{tabular}{C{3cm}|C{5cm}|C{4cm}}
    \hline
    \hline
    Group & Parameter & Value \\
    \hline
    \multirow{6}{*}{Nickel} & Thickness & $30\ \mathrm{nm}$\\
    & H-field & $484.1 \times 10^{3}\ \mathrm{A}/\mathrm{m}$\\
    & Sound velocity $c_{\mathrm{s}}$ & $5.6\times 10^3\ \mathrm{m}/\mathrm{s}$\\
    & Exchange constant $D$ & $430\ \mathrm{meV.A}^2$\\
    & Magnetoelastic constant $b_1$ & $1\times 10^7$\\
    & Electron gyromagnetic ratio $\gamma$& $1.76 \times 10^{11}\ \mathrm{rad}/\mathrm{s.T}$\\
    \hline
    \multirow{3}{*}{Adjustable}& Magnetic field angle $\xi$ & $45\mathrm{\degree}$ \\
    & Acoustic pulse duration & $1\ \mathrm{ps}$ \\
    & Maximal propagation time & $400\ \mathrm{ps}$ \\
    \hline
    \hline
    \end{tabular}
    \caption{Parameter’s value used for the numerical simulation}
    \label{table:parameters}
\end{table}

\newpage

\bibliographystyle{apsrev4-1} % Tell bibtex which bibliography style to use
\bibliography{bib_ExchangeMagnons}

\end{document}
