\documentclass[12pt,a4paper]{article}

%%% REPORT FOR SPIN WAVES EXCITATION BY ACOUSTIC PULSE

\usepackage{amsmath}               %AMSMath tools for flexible equations
\usepackage[english]{babel}
\usepackage{graphicx}
\usepackage{cite}
%\usepackage[margin=1.5cm]{geometry}
\usepackage{fullpage}   % Texte en pleine page, réduction des marges 
\usepackage{subfig} % remplace subfigure
%\usepackage[utf8x]{inputenc} 
\usepackage[latin1]{inputenc} 
\usepackage[T1]{fontenc} 
\usepackage[section]{placeins}
\usepackage{listings} % in order to include code in a tex file
\usepackage{setspace}
\onehalfspacing
\usepackage{hyperref}


%\renewcommand{\thefootnote}{\fnsymbol{footnote}}
\renewcommand{\thefootnote}{\arabic{footnote}}
\DeclareTextSymbol{\degre}{T1}{6}

\newcommand{\valentin}{\textbf{Valentin}}
\newcommand{\vladimir}{\textbf{Vladimir}}
\newcommand{\anton}{\textbf{Anton}}
\newcommand{\alexandr}{\textbf{Alexandr}}
\newcommand{\vasily}{\textbf{Vasily}}
\newcommand{\dmitrii}{\textbf{Dmitrii}}
	
\begin{document}
	
	
\begin{center}
	{\bf\Large Report : Strain dependence on exchange interaction}\\
	{\bf\Large \today}\\
	\vspace{0.4cm}
	{\large Valentin \textsc{Besse}\footnote{Author of the report}, Vladimir \textsc{Vlasov}, Anton \textsc{Golov}, Alexandr \textsc{Alekhin}, Dmitrii \textsc{Kuzmin} and Vasily \textsc{Temnov}}\\
	\vspace{0.6cm}
	%{\large \today}
	%{\large August 28, 2017}
\end{center}
\vspace{0.1cm}

\section*{Version}

\begin{itemize}
    \item February 22, 2018: V1.
\end{itemize}

\section{Introduction}

In this report we discuss the strain dependence of the exchange interaction. 
We present the theory that describe the energy and the field of the exchange interaction (see section \ref{sec1}).
We introduce the strain dependence on these equations and we predict some results.

We conclude by a proposition of the new tasks regarding this specific interaction.

\section{Theory}
\label{sec1}

The energy of the exchange interaction in a ferromagnet can be represented by \cite{gurevich1996magnetization}
\begin{equation}
    U_{\mathrm{ex}} = U_{\mathrm{ex,0}} + U_{\mathrm{ex,NU}}
    \label{eq1}
\end{equation}
where $U_{\textrm{ex,0}}$ is the exchange energy when the magnetization is uniform and $U_{\textrm{ex,NU}}$ is responsible for the change of the exchange energy when the magnetization is non-uniform.
The first term of Eq. \eqref{eq1} can be written
\begin{equation}
    U_{\mathrm{ex,0}} = \frac{1}{2} \mathbf{M} \overleftrightarrow{\Lambda} \mathbf{M}
    \label{eq2}
\end{equation}
with the magnetization $\mathbf{M}$ and the exchange tensor $\overleftrightarrow{\Lambda}$.

The second term of Eq. \eqref{eq1} can be written
\begin{equation}
    U_{\mathrm{ex,NU}} = \frac{1}{2} \sum_{p=1}^3 \sum_{s=1}^3 q_{ps} \frac{\partial \mathbf{M}}{\partial x_p} \frac{\partial \mathbf{M}}{\partial x_s}
    \label{eq3}
\end{equation}
where $q_{ps}$ are the components of a tensor $\overleftrightarrow{q}$.

The effective field is a functional derivative (or variational derivative) of the energy by the magnetization vector
\begin{equation}
    H_{\mathbf{eff}} = - \frac{\delta U}{\delta \mathbf{M}} = - \frac{\partial U}{\partial \mathbf{M}} + \sum_{p=1}^3 \frac{\partial}{\partial x_p} \left[ \frac{\partial U}{\partial \left(\partial \mathbf{M} / \partial x_{p} \right)} \right]
    \label{eq4}
\end{equation}
with the total free energy $U$.
Using Eqs. \eqref{eq3} and \eqref{eq4} we obtain the contribution of the exchange interaction to the effective field
\begin{equation}
    \mathbf{H}_{\mathrm{ex}} = \mathbf{H}_{\mathrm{ex,0}} + \mathbf{H}_{\mathrm{ex,NU}} \equiv \overleftrightarrow{\Lambda} \mathbf{M} + \sum_{p=1}^3 \sum_{s=1}^3 q_{ps} \frac{\partial^2 \mathbf{M}}{\partial x_p \partial x_s}
    \label{eq5}
\end{equation}

So now, let's consider the case of an isotropic ferromagnet, it implies that $\Lambda$ and $q$ are scalars.
We write $q = D$.
We also consider the equation only for the x-direction.
The Eq. \eqref{eq5} becomes
\begin{equation}
    \mathbf{H}_{\mathrm{ex}} = \Lambda \mathbf{M} + D \frac{\partial^2 \mathbf{M}}{\partial x^2}.
    \label{eq6}
\end{equation}
We write the magnetization vector as a Fourier series
\begin{equation}
    \mathbf{M} = \mathbf{M}_0 + \sum_{n=0}^N \mathbf{M}_n \left( t \right) \cos \left( k_n x\right)
    \label{eq7}
\end{equation}
where $\mathbf{M}_0$ is the steady states of the magnetization vector, $\mathbf{M}_n$ is the n-th order of the magnetization vector and $k_n$ is the wavevector of the n-th magnetic order.
It is obvious that $\mathbf{M}_n \ll \mathbf{M}_0$.
Substitutes Eq. \eqref{eq7} into the non-uniform part of Eq. \eqref{eq6} we obtain 
\begin{equation}
    \mathbf{H}_{\mathrm{ex,NU}} =  D \frac{\partial^2 \mathbf{M}}{\partial x} = - D \sum_{n=0}^N k^2_n \mathbf{M}_n^2 \cos \left( k_n x\right).
    \label{eq8}
\end{equation}

The lossless Landau-Lifshitz-Gilbert equation is defined by
\begin{equation}
    \frac{\partial \mathbf{M}}{\partial t} = - \gamma \mathbf{M} \times \mathbf{H}_{\mathrm{eff}}
    \label{eq9}
\end{equation}
with $\gamma$ is the gyromagnetic ratio.
We consider only the contribution of the exchange interaction when the magnetization is non-uniform.
Adding Eq. \eqref{eq8} into Eq. \ref{eq9} we obtain
\begin{equation}
    \frac{\partial \mathbf{M}}{\partial t} = - \gamma D \mathbf{M} \times \frac{\partial^2 \mathbf{M}}{\partial x^2}
    \label{eq10}
\end{equation}
Let's consider that $\mathbf{M} \sim \exp \left( i \omega t \right)$ and that the exchange constant $D$ is modified by the propagation of a strain.
It is easy to imagine that since the distance between two atoms change when a strain is propagating that the exchange interaction.
It implies that $D \left(t,x\right)$.



\section{Conclusion}

\newpage

\section*{List of abbreviations}

\begin{table}[ht]
    %\centering
    \begin{tabular}{ l c r }
        Landau-Lifschitz-Gilbert & $\Longrightarrow$ & LLG \\
        Ferromagnetic resonance & $\Longrightarrow$ & FMR \\
    \end{tabular}
\end{table}

\bibliographystyle{ieeetr}
\bibliography{Exchange_Magnons}

\end{document}